\section{Conceptual problems with time integration in smoothed pressure schemes}
\label{sec:timeintegration}

It is worth noting that all of the above schemes all perform well on
hydrodynamics test suites (see below for more information). However,
when entering the world of cosmological simulations, there are two main things
that must be considered. The first is multiple time-stepping, without
which we would not be able to perform these calculations \citep[see ][]{Borrow2018};
the second is the injection and dissipation of energy from sub-grid
terms. The latter involves two main components: cooling (dealt with here),
and feedback (see \S \ref{sec:energyinjection}).

When cooling on a particle-by-particle basis in a pressure-based scheme, it
is impossible to keep the internal energy and pressure synchronised for
particles that are being drifted. This is an issue in practice because when
an active particle interacts with an inactive particle, the active particle
uses both the internal energy and smooth pressure from the inactive particle
(and from itself) in its equations of motion (see Equations \ref{eqn:pu_dvdt}
and \ref{eqn:pu_dudt}). Importantly, these are non-symmetric errors, i.e. the
error that is made in the velocity equation of motion is not mirrored exactly
by the error made in the internal energy differential, possibly leading to
energy conservation issues.

To highlight the problem, consider two particles. One is currently active (
particle $i$), and neighbours a particle that is currently inactive
(particle $j$). To get the properties of particle $j$ at the current time $t$,
it must be drifted forward to the current time from when it was last kicked.

It does not seem to be possible to construct a time differential for the smoothed
pressure, i.e. $\mathrm{d} \bar{P}_j / \mathrm{d} t$ is unknown. We do know
how to drift a density, however, as this simply depends on the divergence of the
velocity field \citep[see ][]{Price2012}, so if we construct a weighted density
that should follow similar rules for drifting. The following algorithm is followed
to drift the smoothed pressure:
\begin{enumerate}
    \item Calculate the weighted density of the particle
          $\bar{\rho}(t - \Delta t) = \bar{P}(t - \Delta t) / u (t - \Delta t)$
    \item Drift this quantity to construct $\bar{\rho}(t)$
    \item Drift the internal energy using $\mathrm{d} u / \mathrm{d} t$ to
          construct $u(t)$
    \item Reconstruct the weighted pressure $\bar{P}(t) = \bar{\rho}(t) u(t)$.
\end{enumerate}
What we have really done here is assumed that the smoothed pressure can
be drifted with a drift operator $D_{\bar{P}} = D_{\rho} / D_{u}$. This is
problematic, as it assumes that the drift operator for $u$ is the same for
all particles in a given kernel, which under cooling is clearly not true.

The only way out of this is to re-compute the smoothed pressure for all particles
that are interacted with every step, however this would be computationally
infeasible.

The product (at least this is the current conjecture) is that this problem is
what gives rise to the streaks in the phase space diagrams that we have been
seeing in the \swift{} \eagle{} implementation tests. The reason why these
issues are smaller in an entropy based scheme (compared to an energy-based
scheme) is that there is a significantly smaller dynamic range in the
entropy, which is then compounded further by the exponent of $1/\gamma$ that
the entropy enters the equation of motion with. Here, small errors in the
energy lead to negligible errors in the entropy.