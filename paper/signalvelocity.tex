\subsection{Conduction velocity}

There is some freedom in the choice of the conduction velocity
in Equation \ref{eqn:diffusion}. This choice, in a cosmological simulation,
should be able to do the following things:
\begin{itemize}
    \item Allow the diffusion to stabilise contact discontinuities
    \item Reduce diffusion away from contact discontinuities wherever
          possible
    \item Respect a pressure gradient that holds up hydrostatic equilibrium
\end{itemize}
The latter two points are also handled by the evolution of the diffusion
parameter $\alpha_D$ (Equation \ref{eqn:alpha_D}), in that we only attempt
to activate the diffusion when there is a non-linear gradient in the internal
energy.

There have been a number of signal velocities proposed, with the initial
one coming from the signal velocity for the rest of the hydrodynamics,
governed by the soundspeed,
\begin{equation}
    v_{D, ij} = c_i + c_j - \mu_{ij},
    \label{eqn:vdiffsoundspeed}
\end{equation}
which would ensure that the speed of diffusion grows as the gas gets hotter.
The second possibility, discussed by \citep{Wetzley2007}, was chosen to
ensure that hydrostatic equilibrium is respected, is
\begin{equation}
    v_{D, ij} = \left| \frac{\mathbf{v}_{ij} \cdot \mathbf{x}_ij}{r_ij} \right|,
    \label{eqn:vdiffvij}
\end{equation}
This ensures that if there is no net movement of particles locally (i.e. an un-divergent
velocity field) that there will be no diffusion.
Finally, we have the choice proposed by \citep{Price2008}, where
\begin{equation}
    v_{D, ij} = \sqrt{\frac{|P_i - P_j|}{\frac{1}{2}(\rho_i + \rho_j)}}
    \label{eqn:vdiffP}
\end{equation}
that ensures that the diffusion is only ever activated where there is a difference in
pressure across neighbouring particles.
