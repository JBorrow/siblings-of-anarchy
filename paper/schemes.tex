\section{Schemes}

This paper compares several schemes: Density-Energy SPH (DU), Pressure-Energy
SPH (PU), Pressure-Energy SPH with modern switches for both artificial
conduction and viscosity (\anarchy{}-PU), and Density-Energy SPH with modern
switches for artificial conduction and viscosity (\anarchy{}-DU). We pair these
with the quintic spline kernel, as described in \citet{Dehnen2012}, to avoid
the E0 errors that are present in the Wendland kernels. Throughout the paper
we take a ratio of smoothing length to mean inter-particle separation
$\eta=1.238$ to give 48 neighbours with the cubic spline for compatibility
with the well known Gadget-2 cosmological simulation code, for which this
parameter and kernel are the default.

In the below, cosmological scale-factors are omitted for clarity. All calculations
in the hydrodynamics solver, apart from the integration of the viscosity and
diffusion coefficients, are performed in comoving space.

\subsection{Basic SPH schemes}

\subsubsection{Choice of artificial viscosity}

For the two schemes described in this sub-section, DU and PU, we use the
same, fixed, artificial viscosity prescription as described in \citet{Monaghan1992},
along with a Balsara switch \citet{Balsara1989}.

This uses a viscous equation of motion for all particles,
\begin{equation}
    \frac{\mathrm{d}\mathbf{v}_i}{\mathrm{d}t} = 
    - \sum_j \frac{1}{2} \nu_{ij} \left(
        \nabla W(r_{ij}, h_i) + \nabla W(r_{ji}, h_j)
    \right),
    \label{eqn:du_dvdt}
\end{equation}
which is a sum over neighbouring particles $j$, with $h$ the smoothing
length, $m$ the particle mass, $r_{ij}$ the separation between the particles,
and $W$ the SPH kernel. The dimensionless viscosity parameter $\alpha=0.8$.
The viscosity coefficient is defined as follows for each pair of particles
\begin{equation}
    \nu_{ij} = - \frac{1}{2}
        \frac{\alpha (B_i + B_j) \mu_{ij} v_{{\rm sig}, ij}}{\rho_i + \rho_j}.
\end{equation}
The signal velocity for each particle is given by the maximum of $v_{{\rm
sig}, ij} = c_i + c_j - 3\mu_{ij}$\footnote{This represents the usual choice
of $\alpha=1$ and $\beta/\alpha=3$. Note that, unlike \citet{Price2018}, here
we do not explicitly include the $\alpha$ from variable artificial
viscosities in the signal velocity.} over its neighbours with
\begin{equation}
    \mu_{ij} = \begin{cases}
        \frac{\mathbf{v}_{ij} \cdot \mathbf{x}_{ij}}{|\mathbf{x}_{ij}|} & \mathbf{v}_{ij} \cdot \mathbf{x}_{ij} < 0,\\
        0                                                                 & {\rm otherwise}.
    \end{cases}
\end{equation}
Here, $\mathbf{x}_{ij} = \mathbf{x}_i - \mathbf{x}_j$ and $\mathbf{v}_{ij} =
\mathbf{v}_i - \mathbf{v}_j$. The Balsara switch \citet{Balsara1989} is
implemented as follows,
\begin{equation}
    B_i = \frac{|\nabla \cdot \mathbf{v}_i|}{
        |\nabla \cdot \mathbf{v}_i| + |\nabla \times \mathbf{v}_i| + 10^{-4} c_i / h_i
    }
\end{equation}
with the spatial derivatives calculated in the usual way \citep{Price2012}.

Along with this equation of motion, there is an associated equation of motion
for the particle-carried internal energy $u$ (or entropy $A$, with the latter
omitted for brevity),
\begin{equation}
    \frac{\mathrm{d}u_i}{\mathrm{d}t} = 
        \sum_j m_j \frac{1}{2}\nu_{ij} \mathbf{v}_{ij} \cdot
        \overline{\nabla W_{ij}}
\end{equation}
with $\overline{\nabla W_{ij}} = \frac{1}{2}(\nabla W(r_{ij}, h_i) + \nabla W(r_{ji}, h_j))$.

\subsubsection{Density-Energy SPH (DU)}

This implementation of SPH uses the smooth density for each particle $i$,
\begin{equation}
    \rho_i = \sum_j m_j W(r_{ij}, h_i).
\end{equation}
This density is then used with an equation of state, $P_i = (\gamma - 1) u_i \rho_i$,
along with a particle-carried internal energy per unit mass $u_i$, to produce the
following SPH equation of motion:
\begin{align}
    \frac{\mathrm{d}\mathbf{v}_i}{\mathrm{d}t} = 
    - \sum_j m_j &\left[
        f_i \frac{P_i}{\rho_i^2} \nabla W(r_{ij}, h_i) +
        f_j \frac{P_j}{\rho_j^2} \nabla W(r_{ji}, h_j)
   \right],
\end{align}
with the $f$ factors
\begin{equation}
    f_i = \left(1 + \frac{h_i}{n_d \rho_i} \frac{\partial \rho_i}{\partial h_i}\right)^{-1},
\end{equation}
with $n_d$ the number of spatial dimensions, which is introduced to account
for the variable smoothing lengths required for adaptive astrophysical
problems. The derivatives here are calculated in the same way as \citet{Price2012}.

As this scheme tracks energy as the thermal variable, we also have an associated
equation of motion for the internal energy,
\begin{equation}
    \frac{\mathrm{d}u_i}{\mathrm{d}t} = \sum_j m_j f_i \frac{P_i}{\rho_i^2}
    \mathbf{v}_{ij} \cdot \nabla W(r_{ji}, h_j).
\end{equation}

\subsubsection{Pressure-Energy SPH (PU)}

Pressure-Energy SPH differs from the above Density-Energy SPH in that it
relies on an additional smoothed quantity, the smooth pressure
\citep{Hopkins2013},
\begin{equation}
    \bar{P}_i = \sum_j (\gamma - 1) u_j m_j W(r_{ij}, h_i),
    \label{eqn:smoothpressure}
\end{equation}
that enters the equation of motion
\begin{align}
    \frac{\mathrm{d} \mathbf{v}_i}{\mathrm{d} t} = -\sum_j (\gamma - 1)^2 m_j
    u_j u_i 
    \left[ \frac{f_{ij}}{\bar{P}_i} \nabla W(r_{ji}, h_j) +
    \frac{f_{ji}}{\bar{P}_j} \nabla W(r_{ji}, h_j)   \right].
    \label{eqn:pu_dvdt}
\end{align}
Note that we can always construct a smoothed pressure, $\bar{P}$, in any
case. The difference here is that it enters the equation of motion instead of
the un-smoothed pressure $P = (\gamma - 1) u \rho$.

The factors entering the equation of motion due to the variable smoothing
lengths in this case are more complex, with
\begin{equation}
    f_{ij} = 1 - \left[\frac{h_i}{n_d (\gamma - 1) \bar{n}_i m_j u_j}
             \frac{\partial \bar{P}_i}{\partial h_i} \right]
             \left( 1 + \frac{h_i}{n_d \bar{n}_i}
             \frac{\partial \bar{n}_i}{\partial h_i} \right)^{-1}.
\end{equation}
where $\bar{n}$ is the local number density of particles. These factors
also enter the equation of motion for $u$,
\begin{equation}
    \frac{\mathrm{d} u}{\mathrm{d} t} = (\gamma - 1)^2 \sum_j m_j
    u_i u_j \frac{f_ij}{\bar{P}_i} \mathbf{v}_{ij} \cdot \nabla W_{ij}.
    \label{eqn:pu_dudt}
\end{equation}

The sound-speed in Pressure-Energy requires some consideration. To see what
the `correct' sound-speed is, consider the equation of motion (Eq.
\ref{eqn:pu_dvdt}) in contrast to the EoM for Density-Energy SPH (Eq.
\ref{eqn:du_dvdt}). For Density-Energy SPH,
\begin{equation}
  \frac{\mathrm{d}\mathbf{v}_i}{\mathrm{d} t} \sim \frac{c_{s, i}^2}{\rho_i}
\nabla_i W_{ij},
  \nonumber
\end{equation}
and for Pressure-Energy SPH,
\begin{equation}
  \frac{\mathrm{d}\mathbf{v}_i}{\mathrm{d} t} \sim (\gamma - 1)^2
  \frac{u_i u_j}{\bar{P}_i} \nabla_i W_{ij}.
  \nonumber
\end{equation}
From this it is reasonable to assume that the sound-speed, i.e. the speed at
which information propagates in the system through pressure waves, is given by
the expression
\begin{equation}
  c_{s, i} = (\gamma - 1) u_i \sqrt{\gamma \frac{\rho_i}{\bar{P_i}}}.
  \label{eq:sph:pu:soundspeedfromeom}
\end{equation}
This expression is dimensionally consistent with a sound-speed, and includes
the variable used to evolve the system, the smoothed pressure $\bar{P}$.
However, such a sound-speed leads to a considerably \emph{higher} time-step
in front of a shock wave (where the smoothed pressure is higher, but the SPH
density is relatively constant), leading to integration problems. An
alternative to this is to use the smoothed pressure in the place of the
``real" pressure. Whilst it is well understood that $\bar{P}$ should not be
used to replace the real pressure in general, here (in the sound-speed) it is
only used as part of the time-stepping condition. Using the `smoothed' sound-speed
\begin{equation}
  \overline{c_{s, i}} = \sqrt{\gamma \frac{\bar{P}_i}{\rho_i}}
  \label{eq:sph:pu:soundspeed}
\end{equation}
instead of Eq. \ref{eq:sph:pu:soundspeedfromeom} leads to a much improved
time-stepping condition that actually allows particles to be woken up before
being hit by a shock.

\subsubsection{Pressure-Entropy (PA) SPH}

Pressure-Entropy SPH uses a similar set of equations of motion to the
Pressure-Energy case, but instead of tracking the internal energy per unit
mass $u$ as the thermodynamical variable, it instead tracks entropy $A$.
In this scheme we take the smoothed pressure
\begin{align}
    \bar{P}_i = \left[ 
        \sum_j A^{1 / \gamma} m_j W(r_{ij}, h_j)
    \right]^\gamma
\end{align}
which is then evolved in a similar way to the above. This scheme is not
actually fully implemented in \swift{} due to the conceptual issues presented
in \S \ref{sec:energyinjection}. In this case, there is no equation of motion
for $A$ (aside from the diffusive artificial viscosity term), as this is
explicitly conserved under adiabatic expansion and contraction.

\subsection{\anarchy{}-PU SPH}

The \anarchy{} scheme, known for its use in the EAGLE simulations
\citep{Schaye2015, Crain2015, Schaller2015}, builds on top of the
Pressure-Entropy SPH discussed above. It also includes a simplified version
of the artificial viscosity scheme from \citet{Cullen2010}, and an artificial
energy diffusion term \citep{Price2008}. The scheme implemented in \swift{}
differs from the one used in the original EAGLE simulations, using the
internal energy as the thermodynamic variable as opposed to the particle
entropy.

The artificial viscosity scheme is similar to the above, but now the
coefficient $\alpha$ varies with time. The aim here is to have the artificial
viscosity peak at a shock front, and then decay afterwards to reduce spurious
dissipation in non-shocking regions. To that end, we use a pre-shock
indicator
\begin{equation}
    S_i = - h_i^2 \min(\dot{\nabla} \cdot \mathbf{v}_i, 0)
\end{equation}
for each particle. This, the time differential of the velocity divergence,
is calculated implicitly from the value at the previous step, i.e.
\begin{equation}
    \dot{\nabla} \cdot \mathbf{v}_i(t) = \frac{
        \nabla \cdot \mathbf{v}_i(t) - \nabla \cdot \mathbf{v}_i(t - \Delta t)
    }{\Delta t}.
\end{equation}
The velocity divergence is again calculated in the regular SPH way, with no
'special' or 'improved' estimate used. Specifically, we do not use the matrix
calculations presented in \citet{Cullen2010}. The viscosity for each particle
$\alpha_i$ is set using the following logic:
\begin{equation}
    \alpha_i \rightarrow \begin{cases}
       \alpha_{{\rm loc}, i} & {\rm if}~ \alpha_i < \alpha_{{\rm loc}, i}, \\
       \alpha_{{\rm loc}, i} + (\alpha_i - \alpha_{{\rm loc}, i})e^{-\mathrm{d}t/\tau_i} & {\rm if}~ \alpha_i > \alpha_{{\rm loc}, i},\\
       \alpha_{{\rm min}} & {\rm if}~ \alpha_i < \alpha_{\rm min},
    \end{cases}
\end{equation}
with
\begin{equation}
    \alpha_{{\rm loc}, i} = \alpha_{\rm max} S / (S + v_{{\rm sig}_i}^2)
\end{equation}
and 
\begin{equation}
    \tau_i = \ell c_{s, i}/{H_i}$, where $\alpha_{\rm max} = 2.0,
\end{equation}
$\alpha_{\rm min} = 0.0$, $\ell = 0.25$, are dimensionless parameters, and $H_i$
the kernel support radius. Alongside this, we again use the Balsara switch to
reduce dissipation in shear flows; here we take \blue{Josh: check this is implemeted
correctly!}
\begin{equation}
    \nu_{ij} = - \frac{1}{2}
        \frac{(\alpha_i B_i + \alpha_j B_j) \mu_{ij} v_{{\rm sig}, ij}}{\rho_i + \rho_j},
\end{equation}
and otherwise use the same viscous equations of motion as previously.

The final ingredient, which was added to ensure that clusters have flat entropy 
profiles in adiabatic cosmological simulations \citep{Sembolini2016}, and to ensure
that Kelvin-Helmholtz instabilities have a smooth profile, is thermal diffusion. This
thermal diffusion has a switch that aims to ensure that it is only activated in cases
of extreme energy gradients, with
\begin{equation}
    \frac{
        \mathrm{d} \alpha_{D, i}
    }{
        \mathrm{d} t
    } = \beta_D H_i \frac{\nabla^2 u_i}{\sqrt{u_i}} + 
        \frac{\alpha_{D, i} - \alpha_{D, {\rm min}}}{\tau_{D, i}},
\end{equation}
with $\alpha_{D, {\rm min}} = 0$, $\beta_D = 0.01$, and $\tau_{D, i} = c_{s,
i}/{H_i}$ included to allow for decay away from discontinuities. This is then
bounded by $\alpha_{D, max} = 1$, however typical values are of order 0.01,
which is much lower than the typical value in other codes; for instance in
the \phantomsph{} model this $\alpha_D = 1.0$ throughout the simulation
\citep{Price2018}. The actual diffusion is then applied to the particles with
the following equation of motion:
\begin{align}
    \frac{
        \mathrm{d}u_i
    }{
        \mathrm{d}t
    } = 
    \sum_j m_j \frac{\alpha_{D, i} + \alpha_{D, j}}{\rho_i + \rho_j} v_{D, ij} (u_i - u_j) \overline{\nabla W_{ij}}
    \label{eqn:energydiff}
\end{align}
with the diffusion velocity $v_{D, ij} = \max(c_i + c_j + \mu_{ij}, 0)$.
\blue{T\&N comparison TBD}.

\subsection{\anarchy{}-DU}

The \anarchy{}-DU model is the same as the \anarchy{}-PU model, simply swapping out the
Pressure-Energy equations of motion for Density-Energy. This is a step back towards
simplicity (it completely resolves the above discussion about the sound-speed and
signal velocity, for example), as there is no longer a `weighted' quantity entering the
equations of motion.

The major difference in this scheme is that we require different default parameters
for the diffusion scheme. The viscosity scheme is left completely un-touched, but we change
$\beta_D = 0.25$ from the old default of $0.01$ to enable the diffusion to ramp
up significantly faster, and hence reach higher values. This allows $\alpha_D$ to reach
up to the maximum $\alpha_{D, {\rm max}} = 1$ in the following tests.