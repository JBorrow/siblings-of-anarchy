\section{Energy injection in Pressure-Entropy}
\label{sec:energyinjection}

\begin{figure}
    \centering
    \includegraphics{plots/energy_ratio_EAGLE.pdf}\vspace{-0.3cm}
    \includegraphics{plots/energy_ratio.pdf}
    \vspace{-0.7cm}
    \caption{The result of sub-cycling after injecting energy
    in the Pressure-Energy scheme. \emph{Top}: energy injection
    method used in \eagle{}. \emph{Bottom}: best-case scenario
    energy injection.}
    \label{fig:energyinjection}
\end{figure}

For the original \eagle{}/\anarchy{} scheme, Pressure-Entropy was chosen as
the base scheme. This was chosen for compatibility with the existing
Gadget-based code, which tracked entropy as the thermodynamic variable.

This scheme, in regular hydrodynamics tests, performs exactly the same as
the Pressure-Energy scheme presented in this paper. However, when coupling to
sub-grid physics, there are some key differences.

At many points throughout the simulation, we wish to inject energy. For an
entropy-based scheme, clearly some conversion must take place. Considering a
Density-Entropy scheme to begin with \citep{Springel2002}, where we have only
a smooth density $\rho$,
\begin{align}
    P = (\gamma - 1) u \rho,
\end{align}
with $P$ the pressure from the equation of state, $\gamma$ the ratio of specific
heats, and $u$ the particle energy per unit mass. We also have an expression for
the pressure as a function of the entropy $A$,
\begin{align}
    P = A \rho^\gamma.
\end{align}
Given that these should give the same thermodynamic pressure, we can eliminate that
variable and write
\begin{align}
    u = \frac{A \rho^{\gamma - 1}}{\gamma - 1}
\end{align}
and as these variables are independent for a change in energy we can
extract a change in entropy
\begin{align}
    \Delta A = (\gamma - 1)\frac{\Delta u}{\rho^{\gamma - 1}}
\end{align}

Clearly for any energy based scheme (either Density-Energy or
Pressure-Energy), we can simply modify the internal energy per unit mass $u$
of the particles to change the energy of the field. Then, the sum of all energies
in the box will be the original value plus whatever we just chose to inject,
without the requirement for an extra loop over neighbours\footnote{This is only
true given that we do not change the values that enter the smooth quantities,
here the density, at the same time as we inject energy. In practice, the mass
of particles in cosmological simulations either does not change or changes
very slowly with time (due to sub-grid stellar enrichment models).}.

Now considering Pressure-Entropy, we must recall that the smoothed pressure
\begin{align}
    \bar{P}_i = \left[\sum_j m_j A_j^{1 / \gamma} W_{ij} \right]^\gamma.
\end{align}
Note how here the pressure of an individual particle depends on a smoothed
entropy over its neighbours. We can play the game we did for the Density-Energy
scheme again, with
\begin{align}
    \bar{P} = (\gamma - 1) u \bar{\rho} = A \bar{\rho}^\gamma
\end{align}
now being rearranged to eliminate the weighted density $\bar{\rho}$ such that
\begin{align}
    A(u) = \bar{P}^{1 - \gamma} ( \gamma - 1) u^\gamma,
\end{align}
\begin{align}
    u(A) = \frac{A^{1/\gamma} \bar{P}^{1 - 1/\gamma}}{\gamma - 1},
\end{align}
for each particle.

If we wish to inject energy into the \emph{field} by explicitly heating a single
particle, we now need to be careful - notice how when we change the $A$ of one particle,
the smooth pressure $\bar{P}$ of its \emph{neighbours} changes, and hence this
change the internal energy $u$ that they report. When injecting energy, we need
to ensure that we sum up all contributions to the energy of the field, including
the energy that originates from the updated pressures of the neighbours of our
particles.

An ideal algorithm for injecting energy $\Delta u$ in this case would be as
follows:
\begin{enumerate}
    \item Calculate the total energy of all particles that neighbour
          the one which we wish to inject energy into,
          $u_{{\rm field}, i} = \sum_j u(A_j, \bar{P}_j)$.
    \item Find a target energy for the field, $u_{{\rm field}, t} = 
          u_{{\rm field}, i} + \Delta u$
    \item While the energy of the field $u_{{\rm field}} = \sum_j u(A_j, \bar{P}_j)$
          is outside of the bounds of the target energy (tolerance in EAGLE
          was $10^{-6}$) and the number of iterations is below 10,
    \begin{enumerate}
        \item Calculate $A_{\rm inject} = A(u_{{\rm field}, t} - u_{{\rm field}},
              \bar{P})$ for the particle we want to inject into
        \item Add on $A_{\rm inject}$ to the entropy of that particle
        \item Re-calculate the smoothed pressures for all neighbouring particles
        \item Re-calculate the energy of the field $u_{{\rm field}}$
    \end{enumerate}
\end{enumerate}
There is an accompanying \python{} code available in the repository for
this paper that implements this algorithm for a test problem\footnote{}.

This algorithm turns out to be very computationally expensive. Re-calculating
the smoothed pressure for every particle multiple times per step, in the
original EAGLE, was determined to not be possible. An alternative form
of the algorithm was presented that only updates the self contribution
for the individual particles, that follows:
\begin{enumerate}
    \item Calculate the total energy of the particle that we
          wish to inject energy into, $u_{i, {\rm initial}} = u(A_i, \bar{P}_i)$.
    \item Find a target energy for the particle, $u_{i, {\rm target}} = 
          u_{i, {\rm initial}} + \Delta u$
    \item While the energy of the particle $u_i = u(A_i, \bar{P}_i)$
          is outside of the bounds of the target energy (tolerance in EAGLE
          was $10^{-6}$) and the number of iterations is below 10,
    \begin{enumerate}
        \item Calculate $A_{\rm inject} = A(u_{i, t} - u_i,
              \bar{P})$ for the particle we want to inject into
        \item Add on $A_{\rm inject}$ to the entropy of that particle
        \item Update the self contribution to the smoothed pressure for 
              the injection particle by
              $\bar{P}_{i, {\rm new}} = \left[ \bar{P}_{i, {\rm old}}^{1/\gamma} +
              (A_{\rm new}^{1 / \gamma} - A_{\rm old}^{1 / \gamma})W_0 \right]^\gamma$
              with $W_0$ the kernel self-contribution term
        \item Re-calculate the energy of the particle $u_i = u(A_i, \bar{P}_i)$
              using the new entropy and energy of that particle.
    \end{enumerate}
\end{enumerate}
In the EAGLE code, the weighted density was used, however here we have translated
the code to act on smoothed pressures for simplicity. Code is also
provided for the simple test case with this algorithm in the linked repository
\footnote{\url{http://github.com/jborrow/siblings-of-anarchy}}.

The implementations of both of these algorithms are tested in Figure
\ref{fig:energyinjection}, with the ideal algorithm showing reasonable
convergence within the maximal 10 steps that EAGLE requires. Larger relative
errors are made for larger amounts of energy injection; note that a typical
supernovae event in EAGLE heating a particle from $10^{4}$ to $10^{7.5}$ K
(i.e. from the temperature of the warm ISM to the energy criterion set for
feedback) corresponds to a factor of $10^{3.5}$ times the initial energy of
the particle. The initial entropies of the particles were tested as being
equal, or randomised; this did not affect the results.

The algorithm used in EAGLE faces serious problems, leading to significantly
higher than expected energy injection for low energy injection events. For
the case of the requested energy injection being 10 times that of the initial
particle energy, we see a factor of 7 extra energy injected into the field.
As we go to higher levels of energy injection, this error is reduced. As we
inject more entropy, the value $A^{1/\gamma} W_ij$ for all neighbouring
kernels becomes dominant. Then, we calculate the pressure for the central
particle more accurately (as the central contribution dominates) and we
automatically calibrate correctly the fractions of energy from other
particles.

It is important to note, however, that for the case of \eagle{} supernovae,
this should not represent too large of an issue (the energy converges within
10 iterations to around a percent or so). However, these tests are highly
idealised, and only inject energy into a single particle at a time. In a
production run, there may be up to 30\% of the particles in a given kernel
being heated; it is unclear what the effects of heating several particles at
once are. This will be investigated in future work. Also, for feedback
pathways that inject a relatively smaller amount of energy (for instance,
SNIa, or AGN heating particles that have been recently heated and only
allowed to cool a small amount) there will be a significantly larger amount
of energy injected than initially expected.

We are faced with a choice should we wish to use Pressure-Entropy: either we
have to inject the incorrect amount of energy, or we must expend a huge
amount of computational effort re-calculating smoothed pressures. For this reason,
the Pressure-Entropy-based implementation of \anarchy{} (\anarchy{}-PA), was
abandoned in \swift{} and for future \eagle{} projects, in favour of
energy-based schemes.
