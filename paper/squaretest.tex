\begin{figure*}
    \centering
    \includegraphics[width=0.9\textwidth]{plots/SquareTest.pdf}
    \caption{Maps of various quantities in the 2D square test at $t=4$;
    beyond this all schemes are stable. The white dashed line shows the
    initial boundary of the square; this should be a stable configuration. Of
    particular note is the rightmost panel which shows the pressure across
    the medium; blips here will result in artificial surface tension, causing
    the square to slowly morph into a circle. Each plot has an individual
    normalisation for the colour map, as otherwise no structure can be seen
    for any scheme apart from Density-Energy in the pressure panel. The
    structure seen in \anarchy{}-PU scheme is a set of spurious pressure
    waves present since the initial conditions that constantly propagate
    through the medium.}
    \label{fig:squaretest}
\end{figure*}

\begin{figure*}
    \centering
    \includegraphics{plots/SquareTestConvergence.pdf}
    \vspace{-1.0cm}
    \caption{Convergence with resolution for the density in the square test
    (Figure \ref{fig:squaretest}) for the \anarchy{}-DU scheme. Here we see
    that the smooth region created by the diffusion converges towards an
    infinitely small width as a function of resolution, and at high
    resolution this scheme is able to completely capture the expected
    behaviour in this test.}
    \label{fig:squareconvergence}
\end{figure*}

\section{Square Test}
\label{sec:squaretest}


The square test, first presented in \citet{Hopkins2013}, is a particularly
challenging test for schemes that do not use a smoothed pressure in their
equation of motion, as they typically lead to an artificial surface
tension at contact discontinuities (the same ones that lead to the pressure
blip in \S \ref{sec:sodshock}). As the diffusion solved a lot of the problems
at that blip, we can assume that it will go some way to remedying the issue
here. This test is actually a more challenging variant of the ellipsoid test
presented in \citet{Hess2010}, as this includes sharp corners which are
more challenging for diffusion schemes to capture. 

\subsection{Initial conditions}

The initial conditions are generated using non-equal mass particles to ensure
that the kernel evaluations are as symmetric as possible. We set up a grid in
2D space with $n\times n$ particles (in our fiducial run, $n=128$), in a box
of size $L=1$. The central $0.5 \times 0.5$ square is set to have a density
of $\rho_C = 4.0$, with the outer region being set to $\rho_O = 1.0$. The
pressures are set to be equal with $P_C = P_O = 1.0$, with this enforced by
setting the internal energies of the particles to their appropriate values.
All particles are set to be completely stationary in the initial conditions
with $\mathbf{v} = 0$.

\subsection{Results}


Figure \ref{fig:squaretest} shows the results of this test for our various
schemes at $128^2$ resolution. First, we can see that the basic
Density-Energy scheme represents the failure state for this test. It becomes
very round, which is a hallmark of a scheme that includes a surface tension
term; this is shown even more in the pressure plot where the blip from the
Sod shock test has now manifested in a ring of pressure around the high
density region.

The \anarchy{}-DU scheme, with increased diffusion, manages to somewhat rescue
the solution, with only the corners of the square being rounded off before
the diffusion can smooth over the contact discontinuity. This highlights a
problem with initial conditions for the square test with schemes that include
diffusion; to have a stable solution the system first needs to get into a
state where the contact discontinuity has been smoothed over. In realistic
situations, such a discontinuity would always have time to be smoothed as it
formed (e.g. as a cold clump formed from a cooling instability, or as one
blob of gas approaches another). We can see the residual of these initial conditions
issues in the pressure distribution; the initial pressure blip causes sound waves
to continuously propagate through the system and slowly disrupt the square
due to the periodic boundary conditions. The square is also non-viscous, 
so these waves are never damped.

Finally, here we see that even within a pressure-based scheme, there is still
a residual `pressure blip' at the contact discontinuity. The pressure panels
for both of the Pressure-Energy schemes show this, with the residual blip in
the base scheme being significantly stronger than the one present in 
\anarchy{}-DU. The small amount of diffusion in \anarchy{}-PU manages to
somewhat remedy this but has not been able to completely erase it by $t=4$.

In Figure \ref{fig:squareconvergence} we show the convergence with resolution
for the square test in the \anarchy{}-DU scheme. All of the other schemes
do not converge as-such; the Pressure-Energy schemes pass this test at all
resolutions, and the Density-Energy scheme (without diffusion) produces a
circular shape at all resolutions. The \anarchy{}-DU scheme manages to converge
well with resolution, even filling out the corners of the square (and significantly
reducing the size of the transition region) at resolutions of $256^2$ and above.