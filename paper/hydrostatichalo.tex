\section{Hydrostatic Halo}

\begin{figure*}
    \centering
    \includegraphics{plots/hydrostatic_halo.pdf}
    \vspace{-0.7cm}
    \caption{The particle distributions after 10 dynamical times
    of the hydrostatic halo for various thermodynamic quantities, for the
    two \anarchy{} schemes. Note that \anarchy{}-PU has significantly less
    diffusion than the \anarchy{}-DU scheme. The blue background scatter
    shows all particles, with the black dots and their bars representing
    binned quantities and a $1\sigma$ scatter. The purple dashed line shows
    the expected solution, with the smaller panels showing the residual.}
    \label{fig:hydrostatichalo}
\end{figure*}

In this section we consider the effect of the diffusion in a situation where
it appears that we very much want none; a hydrostatic halo that must stay in
equilibrium. This halo has a pressure gradient that is generated from the
gravitational potential.

\subsection{Initial conditions}

The test uses the initial conditions from the \swift{} repository, which
distribute gas particles randomly using rejection sampling in a cube with a
side length of $4$ Mpc. The halo has a virial radius of $0.5$ Mpc. The
density profile of the gas is proportional to $r^{-2}$ where $r$ is the
distance from the centre of the cube. The solution is then plotted after ten
dynamical times.

\subsection{Results}

This test shows how the diffusion scheme is able to minimise differences between
the internal energies of neighbouring particles, in an effort to ensure that scatter
is reduced within a single kernel. This is the same property that enables
the schemes with diffusion to overcome the small blobs that can form in 
the Kelvin-Helmholtz test (\S \ref{sec:kelvinhelmholtz}). Over the run-time of the
simulation, this then becomes a global property, significantly reducing the scatter
of internal energy as a function of radius. The pressure profile is constrained by
the gravitational potential and enforced as we explicitly conserve energy with
the diffusion scheme; the equations ensure that all energy that is removed from 
a particle is given back to another particle in a pairwise manner.

Similar results to the \anarchy{}-PU scheme are seen using the base Pressure-Energy
scheme without diffusion; here the limited $\beta_D = 0.01$ is not enough to equalise
the internal energy across the kernel and reduce the scatter. 