\section{Sod shock tube}
\label{sec:sodshock}

\begin{figure*}
    \centering
    \includegraphics{plots/SodShock.pdf}
    \vspace{-0.7cm}
    \caption{Individual quantities plotted against the analytic solution
    (purple dashed line) for the Sod shock tube. The horizontal axis
    shows the $x$ position of the particles. All particles are shown in blue
    with a sub-set shown in black. All panels are shown at the same
    time $t=0.2$.}
    \label{fig:sodshock}
\end{figure*}

The Sod shock tube is a classic Riemann problem often used to test
hydrodynamics codes. The tube is made up of three main sections in the final
solution: the rarefaction wave ($0.7 < x < 1.0$), contact discontinuity
($x\approx1.2$), and a weak shock ($x\approx1.4$) at the time that we show it
in Figure \ref{fig:sodshock}.

\subsection{Initial Conditions}

The initial conditions for the Sod shock tube use two BCC glass files, one
with a step higher resolution (e.g. the $32^3$ is paired with the $64^3$).
The gas is given an adiabatic index $\gamma = 5/3$, and the two glass files
are set up next to each other to form a discontinuity, with the high
resolution on the left and the low resolution on the right, with density
$\rho_L = 1$, $\rho_R = 1/8$, velocity $v_L = v_R = 0$, and pressure $P_L =
1$, $P_R = 0.1$.

\subsection{Results}

The schemes in general reduce to the analytic solution with very little scatter.
The key exceptions are the Density-Energy scheme which shows the pressure blip
at the contact discontinuity, leading to some scatter in the velocity profile around
this point, and the extreme scatter in velocity shown in the \anarchy{}-DU scheme.
This scatter appears to be a bug at the current time in the implementation as this
is not present in other published results using similar schemes. However, it should
be noted that the pressure blip is significantly suppressed here. We can see how this
occurs by looking at the internal energy profile, where the energy has been smoothed
over the discontinuity.

The smoothing also occurs in the \anarchy{}-PU scheme which has a smaller amount of
diffusion present. Both the pressure schemes, as expected, suppress the pressure
blip.

Another thing to note with the Sod shock is the performance of the various
schemes around the rarefaction wave. The terms implemented in the \anarchy{}-DU
scheme lead to second-order convergence with the smoothing length here,
similar to schemes that include a Riemann solver like GIZMO
\citep{Hopkins2015, Price2018}. However, the smooth pressure schemes both show a
significant overestimation in the pressure at the turnover point. This is
because of the way that \swift{} constrains the smoothing length; particles
find a biased number of neighbours in the pre-expansion region and hence they
contribute more to the kernel than expected. This then leads to an
overestimation of the pressure in this region.