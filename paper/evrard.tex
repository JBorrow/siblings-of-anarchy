
\begin{figure*}
    \centering
    \includegraphics{plots/EvrardCollapsePressure.pdf}
    \vspace{-0.7cm}
    \caption{Pressure as a function of radius from the centre of
    the sphere for the Evrard collapse, shown for our four schemes at time
    $t=0.8$ s. As usual, the blue background shows all particles, with the
    foreground black spots showing binned data with a $1\sigma$ scatter. In
    this case, as there is no analytic solution available for the Evrard, we
    show the solution with a high resolution 1D PPM code as the dashed purple
    line.}
    \label{fig:evrardpressure}
\end{figure*}


\section{Evrard Collapse}

The Evrard collapse \citep{Evrard1988} test takes a large sphere of
self-gravitating gas, at low energy and density, that collapses in on itself,
causing an outward moving accretion shock. This test is of particular interest for
cosmological and astrophysical applications as it allows for the inspection
of the coupling between the gravity and hydrodynamics solver.

\subsection{Initial Conditions}

Gas particles are set up with an adiabatic index of $\gamma=5/3$, mass $M=1$,
radius $R=1$, initial density profile $\rho(r) = 1 / 2\pi r$, and in a very
cold state with $u = 0.05$, with the gravitational constant $G=1$. These
initial conditions are created in a box of size 100, ensuring that there are
no effects from the periodic boundary. Unfortunately, due to the non-uniform
density profile, it is considerably more challenging to provide relaxed
initial conditions (or use a glass file). Here, positions are simply drawn
randomly to produce the required density profile, with the same set of
initial conditions used for all schemes.

\subsection{Results}

In Figure \ref{fig:evrardpressure} we show the pressure distribution of the
collapsing sphere at $t=0.8$ s for all schemes. We see that the pressure distribution
is captured equally well by all schemes, with the shock resolved at a slightly
smaller width for the density-based schemes (as shown in the previous sections).
In particular, we note that the diffusion in the \anarchy{} schemes explicitly does
\emph{not} cause excess diffusion of energy into, or out of, the centre of the
collapsing sphere, with the pressure distribution for all schemes being very
similar in this region.

\begin{figure}
    \centering
    \includegraphics{plots/EvrardEnergy.pdf}
    \vspace{-0.7cm}
    \caption{Energy as a function of time for our various schemes (different
    coloured lines) in the Evrard collapse, split by different components
    (different line styles, see legend). The grey dashed line shows the
    time at which the results are compared to the reference solution
    in Figure \ref{fig:evrardpressure}. See Figure \ref{fig:evrardenergy}
    for a breakdown of how the individual schemes compare to each other.
    All schemes capture similar behaviour, with them all conserving energy
    to within 0.003\%.}
    \label{fig:evrardenergytotal}
\end{figure}

The Evrard collapse is of particular interest because it allows us to
investigate the interplay between the energy captured by the hydrodynamics
(broadly this corresponds to the kinetic energy $E_K$ and thermal energy
$E_U$) and the energy captured by the gravitational dynamics (the potential
energy, $E_P$). Comparisons between these different energies are shown in
Figure \ref{fig:evrardenergytotal}, showing tight correspondance between all
schemes. Note that all schemes conserve \emph{total} energy to within 0.003\%
at all times.

\begin{figure}
    \centering
    \includegraphics{plots/EvrardEnergyRatio.pdf}
    \vspace{-0.4cm}
    \caption{The (hydrodynamical) energy as a function of time in the Evrard
    collapse problem. Due to an absence of an analytic solution, we show all
    schemes with reference to \anarchy{}-DU, with the values representing the
    percentage difference between that scheme and the reference. The grey
    dashed line shows the time at which the results are compared with the
    reference solution in Figure \ref{fig:evrardpressure}. Absolute values
    of the energies are compared, to enable easier comparisons between the 
    potential and thermal energies.}
    \label{fig:evrardenergy}
\end{figure}

To investigate further the differences in energy make-up for the different
schemes, we compare them directly to each other in Figure
\ref{fig:evrardenergy}. We chose to compare against the \anarchy{}-DU model
here as this allowed for the display of the maximal dynamic range in this
figure. Here we see that the schemes that use the \anarchy{} model lead to
lower thermal and kinetic energies, with there being more potential energy
overall. The basic SPH schemes have significantly higher kinetic energies,
particularly around the time when the shock reaches the centre of the sphere
(around $t=1$ s). Importantly, even though we have included thermal
diffusion, the differences between the total kinetic energy in all schemes is
within the 2\% level, implying that the diffusion is not powering spurious
transfer of energy. The overall energy conservation between the schemes, as noted
above, is within the 0.003\% level and does not significantly vary
between schemes.