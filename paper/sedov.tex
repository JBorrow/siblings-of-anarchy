\section{Sedov-Taylor Blastwave}

\begin{figure*}
    \centering
    \includegraphics{plots/SedovBlast.pdf}
    \vspace{-0.7cm}
    \caption{The pressure profile (as a function of radius from the energy
    injection point) for the Sedov-Taylor blastwave, using our various
    schemes. The different panels are all shown at the same time $t=0.05$.
    The blue points show all particles in the simulation volume, with the
    black circles representing binned quantities and their scatter as 
    error bars. The purple dashed line shows the analytic solution.}
    \label{fig:sedov}
\end{figure*}

The Sedov-Taylor blastwave (Sedov blast) \citep{Taylor1950, Sedov1959}
follows the evolution of a strong shock front through an initially isotropic
medium. This is a highly relevant test for cosmological simulations, as this
is similar to the implementations used for sub-grid (below the resolution
scale) feedback from stars and black holes. In SPH schemes this
effectively tests the artificial viscosity scheme for energy conservation; if
the scheme does not conserve energy the shock front will be misplaced.

\subsection{Initial Conditions}

Here, we use a glass file generated by allowing a uniform grid of particles
to settle to a state where the kinetic energy has stabilised. The particle
properties are then initially set such that they represent a gas with
adiabatic index $\gamma = 5/3$, a uniform pressure of $P_0 = 10^{-6}$,
density $\rho_0 = 1$, all in a 3D box of side-length 1. Then, the $n=15$
particles closest to the centre of the box have energy $E_0 = 1/n$ injected
into them.

\subsection{Results}

The pressure as a function of radius in the Sedov blast is shown in Figure
\ref{fig:sedov}. We choose to show the pressure here as it allows for a
simultaneous investigation of the energy (which has a very large dynamic range
in this test) and the density in one panel.

This test highlights the improvement that a more complex artificial viscosity
scheme provides. The Density-Energy and Pressure-Energy schemes (both using a
\citet{Monaghan1992} fixed, low, artificial viscosity parameter). The \anarchy{}
schemes, both using our modified version of the \citet{Cullen2010}, all reach their
saturation point of $\alpha = 2$ for this test in the pre-shock region, allowing
for far improved shock-capturing and significantly reduced particle disorder.
The density-based schemes manage to capture the shock width slightly better than the
pressure-based variants. \blue{Need something here about why this is the case}.

The \anarchy{} schemes are shown with and without diffusion. We see that the
schemes that include the artificial diffusion show reduced particle disorder;
this is due to the diffusion equalising the particle energies within the
kernel as they are hit by the shock, leading to a lower level of dispersion
in the velocities, and hence reduced disorder. We see here the drawbacks of
allowing the diffusion coefficient to reach higher values in the
\anarchy{}-DU scheme, with some energy diffusing \emph{out of} the post-shock
region. Here, the density begins to increase, leading to an overall increased
pressure in that region. However, the interplay between these two variables
ensures that the pressure in the post-shock region stays within 5\% of the
analytical solution. How this interacts with sub-grid cooling remains to
be seen.  It is worth noting that with SPH-ALE schemes like GIZMO-MFM \citep{Hopkins2015}, this
effect is much stronger (i.e. those schemes are significantly more diffusive
in the post-shock region; see \blue{my convergence paper out soon}).

The wave-like pattern seen in the post-shock region for all schemes is an artefact
of the initial conditions; using a lattice-like set of initial conditions would
remedy this issue. \blue{TODO: actually remedy this issue.}