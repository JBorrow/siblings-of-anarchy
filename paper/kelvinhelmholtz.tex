\section{Kelvin-Helmholtz 2D}
\label{sec:kelvinhelmholtz}

\begin{figure*}
    \centering
    \includegraphics{plots/kelvin_helmholtz.pdf}
    \vspace{-0.5cm}
    \caption{Density map of the standard Kelvin-Helmholtz 2D test from the
    \swift{} repository at $t=1.75$ s. See text for details.}
    \label{fig:kelvinhelmholtz}
\end{figure*}

\begin{figure*}
    \centering
    \includegraphics{plots/kelvin_helmholtz_highres.pdf}
    \vspace{-0.5cm}
    \caption{Density projection of a significantly higher resolution (36x as
    many particles) Kelvin-Helmholtz test with the \anarchy{}-DU scheme at
    $t=1.6$ s. Notice how the individual eddies are beginning to break up.
    The diffusion in the scheme prevents the kernel blob-sized artefacts that
    we see even with the pressure based schemes. This particular simulation
    includes some (accidentally) seeded sub-structure due to issues with the
    initial conditions generation code; despite this the diffusion scheme
    manages to keep the eddies stable until they go nonlinear.}
    \label{fig:khhighres}
\end{figure*}

The two dimensional Kelvin-Helmholtz instability is presented below. This test
is a notable variant on the usual Kelvin-Helmholtz test as it includes a
density jump at constant pressure (i.e. yet another contact discontinuity).

\subsection{Initial conditions}

The initial conditions presented here are similar to those in \citet{Price2008}.
This is set up in a periodic box of length $L=1$, with the central band between
$0.25 < y < 0.75$ set to $\rho_C = 2$ and $v_{C, x} = 0.5$, with the outer
region having $\rho_O = 1$ and $v_{O, x} = -0.5$ to set up a shear flow. The pressure
$P_C = P_O = 2.5$ is enforced by setting the internal energies of the (equal mass,
note this difference to the square test) particles. Particles are initially placed
on a grid with equal separations.

We then excite a specific mode of the instability, as in typical SPH simulations
un-seeded instabilities are dominated by noise and are both unpredictable and
unphysical, preventing comparison between schemes.

\subsection{Results}

Images of the Kelvin-Helmholtz test at $256^2$ resolution are presented in
Figure \ref{fig:kelvinhelmholtz} for the four schemes just as the
instability transitions out of the linear regime.

The pressure blip present at contact discontinuities again causes significant
issues for the Density-Energy scheme. The blip prevents any mixing between
the two fluid phases and at later times even causes individual bound blobs to
break off from the eddies.

The Pressure-Energy scheme, as expected, stabilises the equations of motion
over the discontinuity, but does not fully enable mixing of the phases due
to some (small) residual surface tension (seen previously in \S \ref{sec:squaretest}).

Adding diffusion into the mix with the \anarchy{}-DU scheme (note this uses the
Density-Energy equation of motion) allows for the phases to slowly mix and
provides by far the most satisfactory answer of all of the available schemes.
A higher resolution version of the test is presented in Figure \ref{fig:khhighres}
to fully demonstrate the ability of the \anarchy{}-DU scheme to effectively
capture thermal instabilities across a contact discontinuity. Small instabilities
form and begin to mix with the surrounding medium, showing large amounts of
substructure completely unimpeded by the surface tension terms that usually plague
density-based schemes.

The diffusion in \anarchy{}-PU, that is required to prevent small blobs
forming, does not fully manage to ensure the flow remains stable. Recall at
this time the (seeded) instability should remain in the linear regime, and
hence the very unstable edges to the flow are seeded by numerical noise.